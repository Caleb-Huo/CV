% LaTeX

\documentclass[12pt]{amsart} \usepackage{amssymb}

%\textwidth = 460pt 
%\textheight = 9in 
%\hoffset=-54pt \voffset=-40pt

% SIDE MARGINS:
\oddsidemargin 0in \evensidemargin 0in

% VERTICAL SPACING:
%\topmargin -.15in
\topmargin -.4in
\headheight 0in \headsep 0.0in
%\footheight 0.5in
\footskip 0.5in

\pagestyle{plain}
%\pagenumbering{}

% DIMENSION OF TEXT:
\textheight 10in \textwidth 6.5in
%

%\textwidth = 470pt
%\textheight = 700pt
%%\topmargin = 0pt
%%\oddsidemargin = 0pt
%\hoffset = -60pt
%\voffset = -50pt

%\input epsf \def\epsfsize#1#2{0.4#1\relax} \def\nl{\hfil\break}

%\renewcommand{\baselinestretch}{1.2}
%\def\Indent{\hskip .2in}


\title[]{Service Statement}

\begin{document}
\maketitle
\thispagestyle{empty}

In modern society, we enjoy convenient living style since many aspects of living necessities are taken care of by other people.
Hence we can more efficiently concentrate on the things we are interested in.
In the academic community,
statistics serves other interdiscipline fields and makes them thrive.
And in return, the prosperity of other fields reinforce the existence and importance of statistics.  
Therefore I believe service is an important approach to establish long time lasting relationship between partners, make communities function well and unite together.
For myself, I have benefited a lot from other people's service.
It is also my strong will to serve other people.


\section{Professional service experience in statistics}
During my Ph.D period, 
I actively participated in services to help others.
I was the IT manager in my lab, in charge of several linux servers.
I could help other people performing large statistical computing jobs, 
and I was also beneficial from it since I had to learn related techniques durning the process.
I also spent a lot of efforts to serve on teaching.
I was a teaching fellow for a special study class on Bayesian data analysis and will teach advanced R computing in Spring 2017. 
I taught three graduate-level classes as a guest lecturer and served as a teaching assistant multiple times. 
I enjoy interacting with students and helping them understand statistics.
In addition, I have collaborated with biologists in the fields of cancer and psychiatry to analyze a broad range of genomic data.
Providing statistical services to other fields makes me feel energetic and excited about statistics.


\section{Non professional service experience}
I also had many non-professional service experiences in the past.
On May 12, 2008,
Great Wenchuan earthquake occurred in southwestern China measuring at 8 $M_s$.
This catastrophe caused 69,197 people  confirmed death, 374,176 injured, 18,222 missing and 4.8 million people homeless.
I was shocked and grieved about the disaster.
I volunteered to join the team to spread public awareness  and call for donations for the people affected by the earthquake.
During my undergraduate period in physics department,  as a leader, I worked with my classmates and hosted a physics festival to explain physical knowledge, 
hold a physical contest, show off the charm of physics to audiences from other departments. 
In my Ph.D period, I was involved in the Intel International Science and Engineering Fair
(Intel ISEF) in Pittsburgh in May 2015, which is world's largest international pre-college
science competition. The premier global science competition for students provided a forum
for more than 1,700 students from 70 countries, regions, and territories to display their
independent research. I volunteered to serve as a judger (totally 5 judgers) and sought for the best work on statistics side.
Serving the public and helping other people always make me feel happy and satisfied.


\section{Future service}
I will continuously devoted myself to services in the future.
I would be glad to host seminars, journal clubs or any other form of service in my future tenure-tracked environment.
I believe I will also benefit from a well serviced community.
 
\end{document}
