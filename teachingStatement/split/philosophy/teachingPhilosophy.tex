% LaTeX

\documentclass[12pt]{amsart} \usepackage{amssymb}

%\textwidth = 460pt 
%\textheight = 9in 
%\hoffset=-54pt \voffset=-40pt

% SIDE MARGINS:
\oddsidemargin 0in \evensidemargin 0in

% VERTICAL SPACING:
%\topmargin -.15in
\topmargin -.4in
\headheight 0in \headsep 0.0in
%\footheight 0.5in
\footskip 0.5in

\pagestyle{plain}
%\pagenumbering{}

% DIMENSION OF TEXT:
\textheight 10in \textwidth 6.5in
%

%\textwidth = 470pt
%\textheight = 700pt
%%\topmargin = 0pt
%%\oddsidemargin = 0pt
%\hoffset = -60pt
%\voffset = -50pt

%\input epsf \def\epsfsize#1#2{0.4#1\relax} \def\nl{\hfil\break}

%\renewcommand{\baselinestretch}{1.2}
%\def\Indent{\hskip .2in}


\title[]{Teaching Philosophy}

\begin{document}
\maketitle
\thispagestyle{empty}
\section{Teaching philosophy}

In my mind, there are three key elements for effective teaching.
Firstly, I am an advocate for problem-based learning approach and 
I believe students will not think about statistics carefully unless they are engaged in real problems.
Concrete statistics examples and homework are essential for students to nail what they learn.
I would also encourage students to engage in real-world problems and challenges,
where students will thrive on the greater flexibility of using statistics.
This will inspire students to obtain a deeper reflection of the subjects they're studying,
and even their enthusiasm to pursue statistics as a career. 
Secondly, I will make the content well organized and help students make connections.
Often, new knowledge is just a variation of what they learned before.
I will help them to develop a clear structure of general framework and 
articulate  individual contents coherently within the global pricute.
By connecting knowledges,
students can reinforce old knowledges and extend to new knowledges.
To make course contents, examples and exercises closely related,
I will try to take advantage of most advanced tools, (e.g. python notebook or Julia notebook) to enhance material preparation.
Thirdly, I will encourage active interactions in class.
This can make sure students are following closely and thinking carefully about class materials.
I would always encourage students to ask questions and deliver the message very clear that there is no stupid question. 
I believe questions are often common among students and
answering one specific question is actually beneficial to a lot of other students.

\section{Balance between teaching and research}

From my own teaching experience as a teaching fellow in my Ph.D period,
I have to admit that preparing good teaching material is time consuming and it is an important issue to balance between research and teaching.
For lecture materials, I would prepare a general framework of teaching content long way before teaching.
Then I can concentrate on improving and refining the content and presentation shortly before teaching time.
I will consider inviting my colleagues with related expertise to give guest lectures, 
and in return I will volunteer to do the same thing for them.
I will also consider spending one or two lectures on recent research topics related to my own expertise, 
which would provide students a broader view of cutting edge problems in statistics.
I believe teaching and research are not  mutually exclusive.
Teaching will help improve presentation skills as it's similar as presenting at conferences or give a talk at seminars.
Designing and preparing lecture outlines and detailed contents is actually a training process similar to literature review or grant proposal writing.
I have also learned form other instructors that novel ideas would arise each time they taught a class.
Therefore I view teaching as a good complementary to research since it will help me carefully think about research and motivate ideas.


\end{document}
