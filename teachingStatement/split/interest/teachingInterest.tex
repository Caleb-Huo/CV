% LaTeX

\documentclass[12pt]{amsart} \usepackage{amssymb}

%\textwidth = 460pt 
%\textheight = 9in 
%\hoffset=-54pt \voffset=-40pt

% SIDE MARGINS:
\oddsidemargin 0in \evensidemargin 0in

% VERTICAL SPACING:
%\topmargin -.15in
\topmargin -.4in
\headheight 0in \headsep 0.0in
%\footheight 0.5in
\footskip 0.5in

\pagestyle{plain}
%\pagenumbering{}

% DIMENSION OF TEXT:
\textheight 10in \textwidth 6.5in
%

%\textwidth = 470pt
%\textheight = 700pt
%%\topmargin = 0pt
%%\oddsidemargin = 0pt
%\hoffset = -60pt
%\voffset = -50pt

%\input epsf \def\epsfsize#1#2{0.4#1\relax} \def\nl{\hfil\break}

%\renewcommand{\baselinestretch}{1.2}
%\def\Indent{\hskip .2in}


\title[]{Teaching Interest}

\begin{document}
\maketitle
\thispagestyle{empty}

Throughout my primary school to graduate school,
I have met many excellent teachers who inspired me to learn actively and even influenced my career choice. 
I appreciate my past experiences with my teachers 
and I believe teachers play an important role to inspire students' interests and even their entire careers.
I am also excited and eager to be a good teacher to influence other students.

\section{Teaching history}

I was very lucky that I have gradually accumulated many teaching experiences as a graduate student.
My teaching experience started as teaching assistant (TA) for physics experimental lab demonstration 
and for office hour where I could help undergraduate students individually with physics problems.
I was also TA for ``Introductory theories and algorithms for high-throughput genomic data analysis"
where I helped the professor to prepare related materials and to design part of homework.
I gave three guest lectures on reproducible research and parallel computing in R,
which drove me to think about how to prepare an entire lecture and make the students benefit from it.
Recently (Fall 2016), 
I was a teaching fellow (with other lecturers) for a special study class on Bayesian data analysis,
where I focused on teaching from a broader perspective of Bayesian computing and connecting many sampling techniques within the global picture.
In Spring 2017, I will teach a class on Advanced R computing with another Ph.D student.
We designed the entire course, including lecture contents, examples, homework and project topics
under Dr. George Tseng's guidance. 
I learned a lot about how to tailor the content and pace of the material to the interests of students, 
which was an exciting challenge.
\section{Teaching interest}
I have solid training in statistics, machine learning and also physics.
I had a Bachelor's and Master's degree in Physics.
During my Ph.D period,
I spent a great amount of effort learning statistics and machine learning and applied what I learned towards research.
My own research expertise lies in high-throughput data analysis, machine learning, optimization, Bayesian methods, graphical model and statistical computing.
I am confident that I am capable of teaching any basic, advanced or interdisciplinary course related to statistics.


\end{document}
