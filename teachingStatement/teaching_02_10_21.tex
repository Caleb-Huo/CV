% LaTeX

\documentclass[12pt]{amsart} \usepackage{amssymb}

%\textwidth = 460pt 
%\textheight = 9in 
%\hoffset=-54pt \voffset=-40pt

% SIDE MARGINS:
\oddsidemargin 0in \evensidemargin 0in

% VERTICAL SPACING:
%\topmargin -.15in
\topmargin -.4in
\headheight 0in \headsep 0.0in
%\footheight 0.5in
\footskip 0.5in

\pagestyle{plain}
%\pagenumbering{}

% DIMENSION OF TEXT:
\textheight 10in \textwidth 6.5in
%

%\textwidth = 470pt
%\textheight = 700pt
%%\topmargin = 0pt
%%\oddsidemargin = 0pt
%\hoffset = -60pt
%\voffset = -50pt

%\input epsf \def\epsfsize#1#2{0.4#1\relax} \def\nl{\hfil\break}

%\renewcommand{\baselinestretch}{1.2}
%\def\Indent{\hskip .2in}


\title[]{Teaching Statement}

\begin{document}
\maketitle
\thispagestyle{empty}

Throughout my primary school to graduate school,
I have met many excellent teachers who inspired me to learn actively and even influenced my career choice. 
I appreciated my past experiences with my teachers 
and I believe teachers play an important role to inspire students' interests and even entire careers.
I am also excited and eager to be a good teacher to influence other students.

\section{Teaching history}

I was very lucky that I have gradually accumulated many teaching experiences as a graduate student.
My teaching experience started as teaching assistant (TA) for physics experimental lab demonstration 
and for office hour where I could help undergraduate students individually with physics problems.
I was also TA for ``Introductory theories and algorithms for high-throughput genomic data analysis"
where I helped the professor to prepare related materials and think about designing part of homework.
I gave three guest lectures on reproducible research and parallel computing in R,
which drove me to think about how to prepare an entire lecture and make the students benefit from it.
Recently, I was a teaching fellow (with other lecturers) for special study Bayesian data analysis class,
where I try to teach from a broader perspective of Bayesian computing and connect individual sampling technique within general picture.
Next semester, I will teach Advanced R class with another teaching fellow.
We designed the entire course, including lecture topics, audience, homework and projects. 
I learned a lot about how to tailor the content and pace of the material to the expertise and interests of the students, 
which was an exciting challenge to take into account many practical factors.

\section{Teaching Philosophy}

In my mind, there are three key elements for effective teaching.
Firstly, I am an advocate for problem-based learning approach and 
I believe students will not think statistics carefully unless they are engaged in real problem.
Concrete statistics examples and homework are essential for student to nail what they learned.
I would also encourage students to real-world problems and challenges.
Instead of comparatively narrow rubrics defined by textbook and homework, 
students will thrive on the greater flexibility of using statistics
which will inspires students to obtain a deeper reflection of the subjects they're studying,
and even their enthusiasm to pursue statistics as a career. 
Secondly, I will make the content well organized and help students make connections.
Often, new knowledge is just a variation of what they learned before.
I will help them to develop a clear structure of general framework and 
articulate  individual contents coherently within the general framework.
By connecting the knowledges,
the students can reinforce old knowledges and extend to new knowledge.
To make the course content, examples and exercise closely related,
I will try to take advantage of most advanced tools, (e.g. python notebook or Julia notebook) to enhance material preparation.
Thirdly, I will encourage active interactions in class.
This can make sure students are following closely and thinking through carefully about class materials.
I would always encourage students to ask questions.
I will deliver the message very clear that there is no stupid question. 
I believe questions are often shared among students and
answering one specific question is actually beneficial to a lot of other students.
I will also be frank if I get stuck at certain point and don't know how to answer a question.
I will go back to check reference or other colleagues and reply to my students the other day.
There are other important teaching strategies and I am open minded to learn and practice as long as the strategies are effective and beneficial for students.


\section{Teaching interest}
I have solid training in statistics, machine learning and also physics.
I had a Bachelor's and Master's degree in Physics.
During my Ph.D period,
I spent a great amount effort learning statistics and machine learning and applied what I learned towards research.
I am confident that I am capable of teaching any basic, advanced or interdisciplinary course related to statistics.

\section{Balance between teaching and research}

From my own teaching experience,
I have to admit that prepare good teaching is time consuming and it is an important issue to balance between research and teaching.
But in my perspective, teaching is a good oppotunity for me to carefully go through a field.
I have learned form other instructors that novel ideas will arise each time they taught a class.
Therefore I view teaching as a good complementary to research since it will help me carefully thinks about research and motivate ideas when preparing the lecture notes.

\section{Summary}
Given my experience, philosophy and interesting on teaching,
I think teaching is really a helpful and exciting job for me.
It will be my great honor to inspire my students,
watch them applying statistics into other disciplines or other domains of their life,
or even pursue a career in statistics related field. 

\end{document}
