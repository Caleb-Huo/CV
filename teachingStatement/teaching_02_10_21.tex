% LaTeX

\documentclass[12pt]{amsart} \usepackage{amssymb}

%\textwidth = 460pt 
%\textheight = 9in 
%\hoffset=-54pt \voffset=-40pt

% SIDE MARGINS:
\oddsidemargin 0in \evensidemargin 0in

% VERTICAL SPACING:
%\topmargin -.15in
\topmargin -.4in
\headheight 0in \headsep 0.0in
%\footheight 0.5in
\footskip 0.5in

\pagestyle{plain}
%\pagenumbering{}

% DIMENSION OF TEXT:
\textheight 10in \textwidth 6.5in
%

%\textwidth = 470pt
%\textheight = 700pt
%%\topmargin = 0pt
%%\oddsidemargin = 0pt
%\hoffset = -60pt
%\voffset = -50pt

%\input epsf \def\epsfsize#1#2{0.4#1\relax} \def\nl{\hfil\break}

%\renewcommand{\baselinestretch}{1.2}
%\def\Indent{\hskip .2in}


\title[]{Teaching Statement}

\author[]{Zhiguang (Caleb) Huo}

\begin{document}
\maketitle
\thispagestyle{empty}

In my memory,
I have met a lot of excellent teachers from primary school to graduate school who left me with deep impression.
In my primary school, 
Lecturer Ma allowed us to take off the class as long as we finish all the tasks she assigned, 
which motivated me try my best to learn and solve the problems.
In my middle school, 
Lecturer Li would like to award me a piece of chocolate as long as I can solve the difficult math problem she assigned to me.
I will skip some of the inspiring teaching in my high school and college.
At graduate school, I found the most useful thing is to read papers and do homework.

My teaching experience.
My teaching experience trace back to my undergraduate time,
as assistant to the analysis class, I volunteered to go over homework and extra problems before exam.
I have 8 months experience in Department of Physics of University of Pittsburgh,
where my roll is a teaching assistant for the experiment lab, where I taught and demonstrated how to perform experiments.
I also worked in TA offices hours where I could tutor students with physics problems one by one.
When I am in biostatistics, I was very lucky to have multiple teaching opportunities.
I delivered 3 guest lectures in different semesters.
I taught 4 lectures about advanced Bayesian computing.
In the coming spring semester, I will teach 6 lectures on advanced R computing.

All these experiences drive me to think how to teach effectively.
In my mind, the most effective learning approach for students in statistics is to go through the content step by step.
I will use multiple approaches to achieve this goal.
Firstly, I would encourage active interaction in class.
I would always encourage students to ask questions.
The first benefit is that asking question means the student is really think through the problem, which is really good for him.
Second benefit is that questions are often shared among student and a lot of other student will also learn from this process.
I will deliver the message very clear to all student that there is no stupid question. 
If you don't understand, you ask and then you understand. That's all what I care about.
I will always rise questions for student to discuss and this will also promote the student to think about the problem.
Secondly, in the world of statistics, all fancy explanations seem pale in face of a concrete example.
In my own experience, the most effective way to learn is to go through a concrete example.
In my class, I will focus on more example to illustrate what is the problem, what is purpose, what is the solution and why should we appreciate the solution.
I think appropriate homework is the key element for student to completely digest what they learned in class.
Project based learning approach would also be my emphasize.
Regular class usually provide a broad perspective of knowledge.
Project based learning  will motivate students learn deep into certain topics.
Thirdly, as a teacher, it is my responsibility to make the content easy to digest.
My emphasize is I want to make the materials well connected and organized.
In this scenario, students could grasp a general picture about what they are learning.
Also since I try to connect all the information, they would find that new established knowledge is just a variation of what they learned before.
Hence they will feel easy to understand the content.
Another issue I want to admit is that students usually differ by their attitude.
Some students may only want to pass the class or just learn what is necessary.
Some other students may want to pursue statistics as their career and want to learn in more detail.
My proposal is to establish a bonus system by extra credit.
I would have extra credit question and more deep materials for those who would like to pay more effort than regular students.
Regular students won't feel insulted because the extra credit won't affect them to pass the exam or get A.

From my own teaching experience,
I have to admit that prepare good teaching is time consuming and it is an important issue to balance between research and teaching.
First, I view teaching as a good resource for me to carefully go through a field.
Several instructor told me what they will find some novelty each semester they taught the class.
Therefore I view teaching as a good complementary to research since it will help me carefully thinks about research when preparing the lecture notes.
Sticking to research is not fun and teaching and talking to students will make me relief from heavy research.

how to balance research and teaching

The ability to think mathematically is one of the most valuable skills
I teach my students.  Course material is very important, of course,
but in many of my classes there are engineers, biologists,
philosophers, and philologists, as well as math majors, all who will
end up using the subject differently.  It is my experience that
students will not learn to think mathematically (nor learn the
material) unless they engage themselves. Thus one of my chief goals
when teaching is to encourage them to take an active role in the
learning process.

I have found that such encouragement requires a deliberate effort.
Some students, of course, take an active role automatically, but to
many the extent of their responsibilities in this regard is not
intuitively obvious. The methods I use to engender participation tend
to vary by circumstance.  Mathematical discussions
outside of class can be very effective (especially for high level
courses), and I have especially enjoyed leading undergraduate research
groups.  Assigning students to teach the course for a day, asking
questions of the class, or dialoging on practice problems can also be
helpful in certain contexts.  Sometimes group work or class worksheets
are appropriate.  In lower level courses, I have found that suggesting
to the class that they should think something over, pausing the
lecture to give them opportunity to do so, and then asking them to
contribute from their musings works very well.

Another motivational tactic is the careful use of examples.
Mathematics, after all, can be difficult to grasp in its most
axiomatic form, and I have found some students to be uninspired by
completely intangible constructions. Giving examples, of course, takes
on a more nuanced meaning as the mathematical abilities of the
students progress---the examples given in a high level analysis class
will have an entirely different flavor than those in a first course on
probability. In lower courses, I find that the most effective examples
are those that bear on the real world as much as they demonstrate
mathematical structure.

I have also found that incorporating technology, when available and
appropriate, is a good way to keep students interested. For instance,
when properly used, a scientific calculator can have great
instructional value, both as a teaching device, and as a motivator.
Examples of this in calculus are ubiquitous, but it remains true in
other courses. In my linear algebra sections, for example, the
students had no problem row reducing by hand, so using the TI-83 to do
some of this work allowed us to focus on the concepts one row reduces
to elucidate, such as linear independence or invertibility. In my
modern algebra course, computing devices allowed us to explore the RSA
algorithm---we often resorted to {\it Mathematica} when the numbers
got large. Even in high level courses, computer algebra systems such
as {\it Macaulay 2} can contribute to students' understanding.  In fact, I
would argue that the ability to compute is itself beneficial and
interesting to students. Thus I try to help them help develop this
ability while effectively presenting important mathematical concepts.

Finally, I find that students respond well when I effectively express
my genuine concern for their learning. A friendly demeanor, a
comfortable classroom setting (where questions and comments are
welcome), and flexible office hours have worked well to this end. My
students tend to appreciate the time, thought, and energy I invest in
my courses, and I, not surprisingly, have found their appreciation
very rewarding.


\end{document}
