% LaTeX

\documentclass[12pt]{amsart} \usepackage{amssymb}

%\textwidth = 460pt 
%\textheight = 9in 
%\hoffset=-54pt \voffset=-40pt

% SIDE MARGINS:
\oddsidemargin 0in \evensidemargin 0in

% VERTICAL SPACING:
%\topmargin -.15in
\topmargin -.4in
\headheight 0in \headsep 0.0in
%\footheight 0.5in
\footskip 0.5in

\pagestyle{plain}
%\pagenumbering{}

% DIMENSION OF TEXT:
\textheight 10in \textwidth 6.5in
%

%\textwidth = 470pt
%\textheight = 700pt
%%\topmargin = 0pt
%%\oddsidemargin = 0pt
%\hoffset = -60pt
%\voffset = -50pt

%\input epsf \def\epsfsize#1#2{0.4#1\relax} \def\nl{\hfil\break}

%\renewcommand{\baselinestretch}{1.2}
%\def\Indent{\hskip .2in}


\title[]{Teaching Statement}

\begin{document}
\maketitle
\thispagestyle{empty}

Throughout my primary school to graduate school,
I have met many excellent teachers who inspired me to learn actively and even influenced my career choice. 
I appreciate my past experiences with my teachers 
and I believe teachers play an important role to inspire students' interests and even their entire careers.
I am also excited and eager to be a good teacher to influence other students.

\section{Teaching history}

I was very lucky that I have gradually accumulated many teaching experiences as a graduate student.
My teaching experience started as teaching assistant (TA) for physics experimental lab demonstration 
and for office hour where I could help undergraduate students individually with physics problems.
I was also TA for ``Introductory theories and algorithms for high-throughput genomic data analysis"
where I helped the professor to prepare related materials and to design part of homework.
I gave three guest lectures on reproducible research and parallel computing in R,
which drove me to think about how to prepare an entire lecture and make the students benefit from it.
Recently (Fall 2016), 
I was a teaching fellow (with other lecturers) for a special study class on Bayesian data analysis,
where I focused on teaching from a broader perspective of Bayesian computing and connecting many sampling techniques within the global picture.
In Spring 2017, I will teach a class on Advanced R computing with another Ph.D student.
We designed the entire course, including lecture contents, examples, homework and project topics
under Dr. George Tseng's guidance. 
I learned a lot about how to tailor the content and pace of the material to the interests of students, 
which was an exciting challenge.

\section{Teaching philosophy}

In my mind, there are three key elements for effective teaching.
Firstly, I am an advocate for problem-based learning approach and 
I believe students will not think about statistics carefully unless they are engaged in real problems.
Concrete statistics examples and homework are essential for students to nail what they learn.
I would also encourage students to engage in real-world problems and challenges,
where students will thrive on the greater flexibility of using statistics.
This will inspire students to obtain a deeper reflection of the subjects they're studying,
and even their enthusiasm to pursue statistics as a career. 
Secondly, I will make the content well organized and help students make connections.
Often, new knowledge is just a variation of what they learned before.
I will help them to develop a clear structure of general framework and 
articulate  individual contents coherently within the global pricute.
By connecting knowledges,
students can reinforce old knowledges and extend to new knowledges.
To make course contents, examples and exercises closely related,
I will try to take advantage of most advanced tools, (e.g. python notebook or Julia notebook) to enhance material preparation.
Thirdly, I will encourage active interactions in class.
This can make sure students are following closely and thinking carefully about class materials.
I would always encourage students to ask questions and deliver the message very clear that there is no stupid question. 
I believe questions are often common among students and
answering one specific question is actually beneficial to a lot of other students.
\\
\\

\section{Teaching interest}
I have solid training in statistics, machine learning and also physics.
I had a Bachelor's and Master's degree in Physics.
During my Ph.D period,
I spent a great amount of effort learning statistics and machine learning and applied what I learned towards research.
My own research expertise lies in high-throughput data analysis, machine learning, optimization, Bayesian methods, graphical model and statistical computing.
I am confident that I am capable of teaching any basic, advanced or interdisciplinary course related to statistics.

\section{Balance between teaching and research}

From my own teaching experience as a teaching fellow in my Ph.D period,
I have to admit that preparing good teaching material is time consuming and it is an important issue to balance between research and teaching.
For lecture materials, I would prepare a general framework of teaching content long way before teaching.
Then I can concentrate on improving and refining the content and presentation shortly before teaching time.
I will consider inviting my colleagues with related expertise to give guest lectures, 
and in return I will volunteer to do the same thing for them.
I will also consider spending one or two lectures on recent research topics related to my own expertise, 
which would provide students a broader view of cutting edge problems in statistics.
I believe teaching and research are not  mutually exclusive.
Teaching will help improve presentation skills as it's similar as presenting at conferences or give a talk at seminars.
Designing and preparing lecture outlines and detailed contents is actually a training process similar to literature review or grant proposal writing.
I have also learned form other instructors that novel ideas would arise each time they taught a class.
Therefore I view teaching as a good complementary to research since it will help me carefully think about research and motivate ideas.

\section{Summary}
Given my experiences, philosophy and interests on teaching,
I believe teaching is really a helpful and exciting job for me.
It will be my great honor to inspire my students,
watch them applying statistics into other disciplines,
or even pursuing their career in a statistics related field. 

\end{document}
