\documentclass[10pt,stdletter,orderfromtodate]{newlfm}
\usepackage{charter}

\newsavebox{\Luiuc}
\sbox{\Luiuc}{%
	\parbox[b]{1.75in}{%
		\vspace{0.3in}%
		\includegraphics[scale=0.3]
		{pitt_logo.png}%!TEX encoding = UTF-8 Unicode
	}%
}%
\makeletterhead{Uiuc}{\Lheader{\usebox{\Luiuc}}}

\newlfmP{headermarginskip=0pt}
\newlfmP{sigsize=1pt}
\newlfmP{dateskipafter=5pt}
\newlfmP{addrfromphone}
\newlfmP{addrfromemail}
\PhrPhone{Phone}
\PhrEmail{Email}

\lthUiuc

\namefrom{Zhiguang Huo (Caleb)}
\addrfrom{%
\vspace{-1cm}\\
	Department of Biostatistics\\
	Graduate School of Public Health\\
	University of Pittsburgh\\
	130 De Soto Street, Parran Hall 7127\\
	Pittsburgh, PA 15261\\
}
\phonefrom{412-979-0592}
\emailfrom{zhh18@pitt.edu}
\addrto{%
\vspace{-1cm}\\
Faculty Search Committee\\
Department of Something\\
Some University\\
University Address}

\greetto{To Whom It May Concern,}
\closeline{Sincerely,}
\begin{document}
\begin{newlfm}

It's my great pleasure to apply for the position of \ldots{}.
I am a Ph.D student in Department of Biostatistics of University of Pittsburgh
under the supervision of Dr. George Tseng and Dr. YongSeok Park.
I had my Bachelor's and Master's degree in physics.
The position is attracting to me because it fits my research interests and long-term career plan.

My research interest lies in both statistical methodology and application on genomics and bioinformatics.
Nowadays, large amount of genomic data are publicly available and
I have worked on developing data integration methodologies to increase statistical power, interpretability and understanding of disease mechanism. 
I am particularly interested in modeling and optimization for high-dimensional data, Bayesian methods, graphical models and statistical computing,
which are capable of accommodating the high-dimensional nature of genomic data.
I have one statistical methodology paper published in JASA and two others under second round of review in AOAS.

I have collaborated with biologists in the fields of cancer and psychiatry to analyze a broad range of genomic data (e.g. microarray data and sequencing data).
I am eager to seek for new opportunities of collaborations in my future tenure-track research environment, 
and I believe in return collaborations will motivate practical methodological ideas.
However, since genomics and bioinformatics are fast-evolving fields,
I am open minded to expand my research towards new data types (e.g. imaging data, single cell data) in the future.

I was a teaching fellow for Bayesian Data analysis class and will teach advanced R in Spring 2017.
I taught three graduate-level classes as a guest lecturer and served as a teaching assistant multiple times.
I enjoy interacting with students and helping understand statistics.
I view teaching as a good opportunity to organize knowledge structure and motivate new ideas for research. 
Since I have solid training in statistics, machine learning and also physics, 
I am confident that I am capable of teaching any basic, advanced or interdisciplinary course related to statistics.

My dream career is to become a devoted statistician to serve both statistics and biology communities.
My long term goal is to bridge between bioinformatics and statistics/machine learning.


Thank you for your consideration.  
I look forward to hearing from you.

\end{newlfm}
\end{document}

